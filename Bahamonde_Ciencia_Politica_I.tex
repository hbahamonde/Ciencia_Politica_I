%LaTeX Curriculum Vitae Template
%
% Copyright (C) 2004-2009 Jason Blevins <jrblevin@sdf.lonestar.org>
% http://jblevins.org/projects/cv-template/
%
% You may use use this document as a template to create your own CV
% and you may redistribute the source code freely. No attribution is
% required in any resulting documents. I do ask that you please leave
% this notice and the above URL in the source code if you choose to
% redistribute this file.

\documentclass[letterpaper]{article}

\usepackage{hyperref}
\hypersetup{
    bookmarks=true,         % show bookmarks bar?
    unicode=false,          % non-Latin characters in Acrobat’s bookmarks
    pdftoolbar=true,        % show Acrobat’s toolbar?
    pdfmenubar=true,        % show Acrobat’s menu?
    pdffitwindow=true,     % window fit to page when opened
    pdfstartview={FitH},    % fits the width of the page to the window
    pdftitle={My title},    % title
    pdfauthor={Author},     % author
    pdfsubject={Subject},   % subject of the document
    pdfcreator={Creator},   % creator of the document
    pdfproducer={Producer}, % producer of the document
    pdfkeywords={keyword1} {key2} {key3}, % list of keywords
    pdfnewwindow=true,      % links in new window
    colorlinks=true,       % false: boxed links; true: colored links
    linkcolor=blue,          % color of internal links (change box color with linkbordercolor)
    citecolor=blue,        % color of links to bibliography
    filecolor=blue,      % color of file links
    urlcolor=blue           % color of external links
}



\usepackage{geometry}
\usepackage{import} % To import email.
\usepackage{marvosym} % face package
\usepackage{xcolor,color}
\usepackage{fontawesome}
\usepackage{amssymb} % for bigstar
\usepackage{epigraph}

% Comment the following lines to use the default Computer Modern font
% instead of the Palatino font provided by the mathpazo package.
% Remove the 'osf' bit if you don't like the old style figures.
\usepackage[T1]{fontenc}
\usepackage[sc,osf]{mathpazo}

% Set your name here
\def\name{Ciencia Pol\'itica I - APU3601}

% Replace this with a link to your CV if you like, or set it empty
% (as in \def\footerlink{}) to remove the link in the footer:
\def\footerlink{}
% \href{http://www.hectorbahamonde.com}{www.HectorBahamonde.com}

% The following metadata will show up in the PDF properties
\hypersetup{
  colorlinks = true,
  urlcolor = blue,
  pdfauthor = {\name},
  pdfkeywords = {intro to social sciences},
  pdftitle = {\name: Syllabus},
  pdfsubject = {Syllabus},
  pdfpagemode = UseNone
}

\geometry{
  body={6.5in, 8.5in},
  left=1.0in,
  top=1.25in
}

% Customize page headers
\pagestyle{myheadings}
\markright{{\tiny \name}}
\thispagestyle{empty}

% Custom section fonts
\usepackage{sectsty}
\sectionfont{\rmfamily\mdseries\Large}
\subsectionfont{\rmfamily\mdseries\itshape\large}

% Don't indent paragraphs.
\setlength\parindent{0em}

% Make lists without bullets
\renewenvironment{itemize}{
  \begin{list}{}{
    \setlength{\leftmargin}{1.5em}
  }
}{
  \end{list}
}


% email input begin
\newread\fid
\newcommand{\readfile}[1]% #1 = filename
{\bgroup
  \endlinechar=-1
  \openin\fid=#1
  \read\fid to\filetext
  \loop\ifx\empty\filetext\relax% skip over comments
    \read\fid to\filetext
  \repeat
  \closein\fid
  \global\let\filetext=\filetext
\egroup}
\readfile{/Users/hectorbahamonde/Bibliografia_PoliSci/email.txt}
% email input end


%%% bib begin
\usepackage[american]{babel}
\usepackage{csquotes}
%\usepackage[style=chicago-authordate,doi=false,isbn=false,url=false,eprint=false]{biblatex}

\usepackage[authordate,isbn=false,doi=false,url=false,eprint=false]{biblatex-chicago}
\DeclareFieldFormat[article]{title}{\mkbibquote{#1}} % make article titles in quotes
\DeclareFieldFormat[thesis]{title}{\mkbibemph{#1}} % make theses italics

\AtEveryBibitem{\clearfield{month}}
\AtEveryCitekey{\clearfield{month}}

\addbibresource{/Users/hectorbahamonde/Bibliografia_PoliSci/library.bib} 
\addbibresource{/Users/hectorbahamonde/Bibliografia_PoliSci/Bahamonde_BibTex2013.bib} 

% USAGES
%% use \textcite to cite normal
%% \parencite to cite in parentheses
%% \footcite to cite in footnote
%% the default can be modified in autocite=FOO, footnote, for ex. 
%%% bib end




\begin{document}

% Place name at left
%{\huge \name}

% Alternatively, print name centered and bold:
\centerline{\huge \bf \name}

\epigraph{\emph{``Man is a political animal. A man who lives alone is either a Beast or a God''}}{Aristotle, Politics}


\vspace{0.25in}

\begin{minipage}{0.45\linewidth}
 Universidad de O$'$Higgins \\
  Instituto de Ciencias Sociales \\
  Rancagua, Chile\\
  \\
  \\

\end{minipage}
\hspace{4cm}\begin{minipage}{0.45\linewidth}
  \begin{tabular}{ll}
{\bf \'Ultima actualizaci\'on}: \today. \\
 {\bf Descarga la \'ultima versi\'on} \href{https://github.com/hbahamonde/Ciencia_Politica_I/raw/master/Bahamonde_Ciencia_Politica_I.pdf}{aqu\'i}.%\\
   %{\bf {\color{red}{\scriptsize Not intended as a definitive version}}} %\\
    \\
    \\
    \\
    \\
    \\
  \end{tabular}
\end{minipage}



\subsection*{Aspectos Log\'isticos}


\vspace{0.8mm}
{\bf Profesor}: H\'ector Bahamonde, PhD.\\
\texttt{e:}\href{mailto:hector.bahamonde@uoh.cl}{\texttt{hector.bahamonde@uoh.cl}}\\
\texttt{w:}\href{http://www.hectorbahamonde.com}{\texttt{www.hectorbahamonde.com}}\\
\texttt{Zoom ID:} \href{https://us02web.zoom.us/j/9513261038?pwd=S3BSWXQxZW11NC9CRjRoMmd0TkpEZz09}{\texttt{951-326-1038}}.\\
{\bf Office Hours (Zoom)}: Toma una hora \href{https://calendly.com/bahamonde/officehours}{\texttt{aqu\'i}}.


\vspace{5mm}
{\bf Hora de c\'atedra}: Martes (10.15-11.45) y jueves (10.15-11.45).\\
{\bf Lugar de c\'atedra}: Zoom (no hay clases presenciales este semestre).\\
{\bf Link Zoom}: \href{https://us02web.zoom.us/j/9513261038}{\texttt{https://us02web.zoom.us/j/9513261038}}.\\
{\bf Acceso a materiales del curso}: \href{https://ucampus.uoh.cl/uoh/2021/1/APU3601/1/}{\texttt{uCampus}}.

\vspace{5mm}
{\bf Ayudante de c\'atedra (TA)}: Gonzalo Barr\'ia.\\
\texttt{e:}\href{mailto:gonzalo.barria@uoh.cl}{\texttt{gonzalo.barria@uoh.cl}}\\
\texttt{Zoom ID:} 988-891-7227.\\
{\bf TA Bio}: Cientista Pol\'itico (PUC) y Mag\'ister en Ciencia Pol\'itica (PUC).\\
{\bf Hora de ayudant\'ia}: {\color{red}PENDIENTE}.\\
{\bf Lugar de ayudant\'ia}: Zoom (no hay ayudant\'ias presenciales este semestre).


\vspace{5mm}
{\bf Carrera}:	Administraci\'on P\'ublica.\\
{\bf Eje de Formaci\'on}: L\'inea Ciencia Pol\'itica.\\
{\bf Semestre/A\~no}:	Quinto Semestre/2020.\\
{\bf Pre-requisitos}: M\'etodos de Investigaci\'on.\\
{\bf SCT}: 6.\\
{\bf Horas semanales}: C\'atedra (3 horas), Ayudant\'ia	(1).



\subsection*{Motivaci\'on: ¿Por qu\'e tomar este curso?}

La democracia y el aparato p\'ublico (el Estado) son sin duda los ``artefactos sociales'' m\'as importantes del mundo moderno. \emph{¿Cu\'ales son los or\'igenes de ambos? ¿Qu\'e tipos de conflictos sociales generan o retardan el progreso de las sociedades? ¿Por qu\'e algunos pa\'ises tienen sistemas multipartidistas mientras que otros tienen s\'olo dos partidos pol\'iticos? ¿Por qu\'e existen los partidos pol\'iticos? ¿Por qu\'e algunos dictadores gobiernan toda la vida, mientras que otros son asesinados, durando s\'olo unos cuantos meses en el poder? ¿Hay alguna diferencia entre un primer ministro y un presidente? ¿D\'onde se vive mejor, en una dictadura o en una democracia? ¿Por qu\'e los pa\'ises tienen los sistemas electorales que tienen?} 
\\
\\
Todos estos, y otros temas, son los que estudia la ciencia pol\'itica. En este semestre, m\'as que ideas sueltas, aprenderemos los {\bf debates} que mira esta disciplina. Por eso es que hemos seleccionado los textos y teor\'ias casi siempre en un formato binario: para cada idea, casi siempre habr\'a o una cr\'itica, o una idea contraria. Lo interesante: ambas casi siempre suenan coherentes. Ser\'a \emph{tu} tarea tomar una posic\'on, y ``resolver'' estos conflictos en los ensayos que deber\'as escribir durante este semestre. 
\\
\\
Finalmente, un$@$ administrador$@$ p\'ublic$@$ no puede considerarse tal si es que no conoce, por ejemplo, el debate acerca del origen del estado moderno, o las consecuencias sociales y econ\'omicas de vivir en democracia (relativo a vivir en dictadura). Es por esto que este curso, espero, te cause gran inter\'es.
\\
\\
\emph{Bienvenid$@$!}


\subsection*{\'Ambitos de Desempe\~no}

\begin{enumerate}
  \item La gesti\'on estrat\'egico-operativa de organizaciones p\'ublicas (estatales y no estatales).
  \item La gesti\'on pol\'itico-estrat\'egica del entorno (regional/nacional).
  \item La participaci\'on, colaboraci\'on e influencia en el proceso de pol\'iticas p\'ublicas.
\end{enumerate}


\subsection*{Competencias y Sub-competencias a las que Contribuye el Curso}

\begin{enumerate}
  \item Define, analiza e interpreta el fen\'omeno organizativo u otro relevante en el que se desenvuelve, utilizando enfoques interdisciplinarios para problematizarlo desde la especificidad de los asuntos p\'ublicos.
     \begin{enumerate} 
      \item Identifica y analiza relaciones, influencias y din\'amicas de interacci\'on entre su organizaci\'on y su entorno, utilizando y conjugando modelos y aproximaciones te\'oricas, enmarcando este proceso, con miradas que incorporan criterios locales regionales y nacionales. 
      \item Construye modelos orientados a interpretar fen\'omenos propios de lo p\'ublico en el entorno local, regional y/o nacional, apoy\'andose en saberes cient\'ificos, reconociendo su rol como agente de transformaci\'on de la realidad.
    \end{enumerate}
  
  \begin{enumerate}
    \item Identifica, diagnostica, analiza y define problemas p\'ublicos relevantes para su entorno local y/o regional desde una perspectiva interdisciplinaria, reconociendo variables que influyen en su naturaleza y resoluci\'on. 
    \item Reconoce e interpreta la relaci\'on entre Estado, pol\'itica, poder, gesti\'on pol\'itica y gesti\'on p\'ublica, desde paradigmas y marcos te\'oricos apropiados, estableciendo patrones de correlaci\'on e influencia entre estos fen\'omenos.
    \item Identifica e interpreta las din\'amicas asociadas al problema p\'ublico utilizando herramientas de an\'alisis situacional y prospectivo, apoyando su an\'alisis en criterios \'eticos. 
  \end{enumerate}

\end{enumerate}





\subsection*{Prop\'osito General del Curso}

Los problemas p\'ublicos han sido entendidos y definidos como resultado de los principales debates pol\'iticos y filos\'oficos a lo largo de la historia de Occidente. Este curso pondr\'a \'enfasis en que l$@$s estudiantes puedan reconocer aquellos aportes e ideas que mayor impacto han tenido en el debate politol\'ogico contempor\'aneo, con el objetivo que comprendan, analicen e interpreten las principales ideas que han estado presentes en el debate te\'orico pol\'itico respecto a los significados subyacentes a la acci\'on pol\'itica y los apliquen en contextos propios de la gesti\'on p\'ublica.

\subsection*{Objetivos Generales del Curso}

El gran objetivo de este curso, es poder cubrir los grandes temas de la ciencia pol\'itica, especialmente, los que est\'an relacionados a las ciencias administrativas y el mundo del Estado. Es por estos motivos que el curso est\'a dividido en tres grandes unidades. Especial atenci\'on se ha puesto en abordar los debates m\'as cl\'asicos, pero al mismo tiempo, m\'as actualizados de la disciplina.


\begin{enumerate}
	\item Democracia.
	\item Autoritarismo.
	\item Estado.
\end{enumerate}
 

\subsection*{Objetivos Espec\'ificos del Curso}

\begin{enumerate}
	\item Mapear los debates m\'as importantes de la ciencia pol\'itica.
	\item Dar la oportunidad de abordar la disciplina desde un punto de vista cr\'itico.
	\item Conocer las distintas perspectivas, muchas veces contradictorias, de nuestra disciplina.
\end{enumerate}


\begin{enumerate}
		\item[{\color{red}\Pointinghand}] Se espera que l$@$s estudiantes hagan sus respectivas lecturas \emph{antes} de cada clase para poder participar en el debate cr\'itico que haremos en cada una de ellas. 

		\item[{\color{red}\Pointinghand}] Este curso tiene bastante lectura. Sin embargo, se ha prestado especial cuidado a que se cumplan todos los reglamentos relacionados al sistema de cr\'editos universitarios. En ayudant\'ia (y en clases) se abordar\'an distintas t\'ecnicas para leer eficientemente.
\end{enumerate}


\subsection*{Resultados de Aprendizaje}

Al final del curso, los/las estudiantes deber\'an ser capaces de,

\begin{enumerate}
  \item Mapear los debates m\'as importantes de la ciencia pol\'itica.
  \item Abordar la disciplina desde un punto de vista cr\'itico.
  \item Conocer las distintas perspectivas, muchas veces contradictorias, de nuestra disciplina.
\end{enumerate}

\subsection*{Competencias Transversales}


\begin{enumerate}
	\item Utiliza y aplica un pensamiento hol\'istico, cr\'itico, l\'ogico y creativo para comprender y explicar los fen\'omenos propios de su entorno.
	\item Desarrolla su labor con apego al Estado de Derecho y la institucionalidad democr\'atica, guiado por los principios de transparencia, imparcialidad, eficacia, eficiencia, probidad, responsabilidad. 
	\item Incorpora la tecnolog\'ia y aplica t\'ecnicas y herramientas apropiadas para la comprensi\'on, an\'alisis y resoluci\'on de problemas p\'ublicos.
\end{enumerate}


\subsection*{Integridad Acad\'emica}

En este curso {\bf el plagio y la copia est\'an absolutamente prohibidos}. Nuestra casa de estudios aun no cuenta con un protocolo de integridad acad\'emica. Mientras tanto, seguiremos las reglas de conducta de  \href{www.extension.harvard.edu/resources-policies/student-conduct/academic-integrity}{\texttt{Harvard University}}. Tod$@$s l$@$s estudiantes de este curso deber\'an familiarizarse con estas reglas y definiciones desde la primera clase. Aunque de manera no exhaustiva, el plagio se define como \emph{hacer pasar por propias las ideas de otros}. De igual manera, la copia se define como \emph{copiar las ideas de otros}. 


En caso de detectarse situaciones de plagio o copia por parte de las/os estudiantes, se pondr\'a la nota m\'inima en la actividad en la que esa situaci\'on ocurri\'o. Adem\'as se informar\'a a la/el jefa/e de carrera qui\'en en conjunto con la Direcci\'on de la Escuela decidir\'an las sanciones que se le aplicar\'an a la/el estudiante, en funci\'on de la gravedad de los hechos detectados.

	
	\begin{enumerate}
		\item[{\color{red}\Pointinghand}] Aunque t\'u ser\'as \emph{absolutamente} responsable de c\'omo usar referencias bibliogr\'aficas apropiadamente, el TA y yo cubriremos con el curso un set de buenas pr\'acticas que te ayudar\'an a no caer en un plagio por omisi\'on. Mientras tanto, puedes \href{https://www.extension.harvard.edu/resources-policies/resources/avoiding-plagiarism}{\texttt{consultar}} los siguientes consejos.
	\end{enumerate}




\subsection*{Metodolog\'ias}

Clases tipo seminario v\'ia Zoom.

\subsection*{Requisitos de Aprobaci\'on y Evaluaciones del Curso}

\begin{enumerate}

	% Participation
	\item {\bf Lecturas y Participaci\'on}: 15\%.

		El TA y yo asumiremos durante todo el semestre que has le\'ido. Nosotros empleamos un m\'etodo de clases interactivo, pero este m\'etodo necesita de tu participaci\'on activa en clases.
		%\\
		%\\	
		%Para asegurarnos de que est\'es haciendo las lecturas, habr\'an una serie de \emph{pop quizzes} (``pruebas sorpresa'') tanto en c\'atedra como ayudant\'ia. Estos controles ser\'an cortos (5-10 minutos), y apuntan a medir si leyeron; o sabes, o no sabes. Estas pruebas se aplicar\'an completamente al azar, en cualquier momento de la clase, y sin previo aviso. En general, las preguntas ser\'an acerca de un concepto clave, y cuya respuesta correcta ser\'a una l\'inea (o dos, como m\'aximo).

	\item {\bf Midterm}: 25\%.

		Al terminar la segundad unidad (``Democracia''), habr\'a un ensayo. De un set de tres preguntas, escoger\'as una, y tendr\'as dos semanas para desarrollar tu respuesta. Entregas atrasadas tendr\'an un 1 autom\'aticamente, sin excepciones. Todos los textos, discusiones y ayudant\'ias podr\'ian ser consideradas. Las preguntas ser\'an relacionales. Lo que gu\'ia la pregunta (y la respuesta) es un tema. No un texto en particular. Es individual. Llegado el momento, se discuntir\'an los pormenores en clases y ayudant\'ia. Preoc\'upate de usar referencias correctamente. S\'olo podr\'as referenciar el material cubierto en este curso. Cuando comiences a prepararte, el ayudante y el profesor estar\'an disponibles para responder preguntas (e-mail y video-conferencia). La extensi\'on deber\'a ser\'a entre 1500 y 2000 palabras. Aseg\'urate de ocupar citas. Recibir\'as instrucciones detalladas en el mismo temario.

	\item {\bf Ensayo Final (obligatorio; no eximible)}: 30\%. 

	Durante la \'ultima clase se entregar\'a un temario de preguntas (de nuevo, enfocadas en temas m\'as que textos particulares). Tendr\'as que escoger una, y desarrollarla en formato ensayo \emph{in extenso}. El plazo de entrega es de dos semanas despu\'es de la \'ultima clase. Entra todo lo visto durante el semestre, y es individual. Deber\'as entregar el ensayo en \texttt{uCampus}. Entregas atrasadas tendr\'an un 1 autom\'aticamente, sin excepciones. Preoc\'upate de usar referencias correctamente. S\'olo podr\'as referenciar el material cubierto en este curso. Cuando comiences a prepararte, el ayudante y el profesor estar\'an disponibles para responder preguntas (e-mail y video-conferencia). La extensi\'on deber\'a ser\'a entre 1500 y 2000 palabras. Aseg\'urate de ocupar citas. Recibir\'as instrucciones detalladas en el mismo temario.





	\item {\bf Exposiciones individuales}: dos en total, 15\% cada una, 30\% en total.

	Durante el semestre, tendr\'as que presentar en no m\'as de 10-12 minutos (pero nunca en menos de 5-8 minutos), y \underline{al comienzo de la clase}, uno de los textos que toca leer ese d\'ia. (Obviamente, ese d\'ia no podr\'as llegar atrasad$@$). Deber\'as estar preparad$@$ para responder preguntas del profesor y la audiencia. Deber\'as inscribir tus dos textos en \texttt{uCampus} al comienzo del semestre. {\bf S\'olo hay una regla de asignaci\'on: el/la que llega primero/a, se queda con el texto}. {\bf S\'olo hay un/a expositor/a por texto; esto implica que pueden haber hasta un m\'aximo de dos presentaciones por clase}. Las exposiciones comienzan con el texto de la segunda clase. No son necesarios los \emph{slides} (``Power Point''). No es necesario que prendas tu c\'amara.

	\underline{Enf\'ocate en lo siguiente}:

		\begin{enumerate}
			\item \emph{¿Cu\'al es el argumento?} Por ejemplo, ``De acuerdo al texto, X causa Y''. {\bf Esta porci\'on de tu exposici\'on es en lo que se debe ocupar m\'as tiempo}. Los trabajos que leeremos son de la m\'as alta calidad. En consecuencia, espera encontrar un argumento sumamente l\'ogico. Deber\'as explicarlo ``paso a paso''. Sin embargo, ning\'un argumento es perfecto, y ser\'a nuestra tarea (partiendo con tu exposici\'on) analizarlos \underline{cr\'iticamente}. 
			\item \emph{¿Cu\'al es la evidencia emp\'irica?} (si es que hay).
			\item \emph{¿Qu\'e es lo que m\'as te convenci\'o/gust\'o del texto?} Aqu\'i lo ``cosm\'etico'' \emph{no} es lo importante. Siempre c\'entrate en \emph{el argumento}.
			\item \emph{¿Qu\'e es lo que menos te convenci\'o/gust\'o del texto?} Aqu\'i lo ``cosm\'etico'' \emph{no} es lo importante. Siempre c\'entrate en \emph{el argumento}.
			\item \emph{¿Se te ocurre un ejemplo o aplicaci\'on de la teor\'ia que le\'iste y presentaste?} Por ejemplo, ¿crees que la teor\'ia funciona bien en Europa pero no en Latino Am\'erica?
			\item \emph{¿Se te ocurre alg\'un cruce/contraste con alg\'un otro texto \underline{del curso}?} Por ejemplo, ¿encuentras que el texto que presentaste se contradice/parece a otro de los textos (incluyendo el otro texto asignado para el d\'ia de tu presentaci\'on)?
		\end{enumerate}

\end{enumerate}


\underline{En resumen}:

\begin{table}[h]
\begin{tabular}{ccc}
							& \textbf{Porcentaje} & {\bf Porcentaje Acumulado} \\
							\hline
Participaci\'on y \emph{pop-quizzes} (c\'atedra y ayudant\'ia) & 15\%       & 15\%                 \\
\hline
Midterm 				& 25\% & 40\%                 \\
Examen final 		& 30\% & 80\%                 \\
\hline
Exposici\'on Individual \#1 	& 15\% & 90\%    \\
Exposici\'on Individual \#2 	& 15\% & 100\%    \\
\hline             
\end{tabular}
\end{table}

\subsection*{Ayudant\'ia}

Cada semana te reunir\'as con tu ayudante (``TA'') v\'ia Zoom. Ah\'i tendr\'as otra oportunidad para ejercitar y seguir profundizando otras tem\'aticas pendientes. En esta oportunidad, tambi\'en se revisar\'an aspectos m\'as formales de las humanidades y las ciencias sociales. 


\subsection*{Calendario}



\begin{enumerate}  
\addtocounter{enumi}{-1}
		\item \underline{Introducci\'on}
					\begin{enumerate}
						\item[1.] \underline{Introducciones, programa, expectativas}
							\begin{enumerate}
								\item No hay lecturas.
							\end{enumerate}
					\end{enumerate}

			\begin{enumerate}
				
		\item[2.] \underline{¿Qu\'e es la ciencia pol\'itica?}
					\begin{enumerate}
						\item Grigsby, Ellen. 2011. \href{https://github.com/hbahamonde/Ciencia_Politica_I/raw/master/Readings/Grigsby.pdf}{\emph{Analyzing Politics: An Introduction to Political Science}}. Wadsworth Publishing. Ch. 2.\phantom{\textcite{Grigsby2011}}
						\item Gary King, Robert Keohane, Sidney Verba. \href{https://github.com/hbahamonde/Ciencia_Politica_I/raw/master/Readings/kkv.pdf}{\emph{Designing Social Inquiry: Scientific Inference in Qualitative Research}}, pp. 3-49. Princeton University Press, 1994.\phantom{\textcite{King1994}}
					\end{enumerate}
			\end{enumerate}
		\item {\bf \underline{Democracia}}
			\begin{enumerate}
			
				\item[3.] {\bf ¿Qu\'e es la Democracia? Teor\'ia y Pr\'actica (consenso y ``Westminster'')}.
					\begin{enumerate}
						\item Lijphart, Arend. 1999. \href{https://github.com/hbahamonde/Ciencia_Politica_I/raw/master/Readings/Lijphart.pdf}{\emph{Patterns of Democracy}}. Yale University Press: Ch. 1---4.\phantom{\textcite{Lijphart2012}}
						\item Dahl, Robert. 1971. \href{https://github.com/hbahamonde/Ciencia_Politica_I/raw/master/Readings/Polyarchy.pdf}{\emph{Polyarchy: Participation and Opposition}}. Yale University Press. Ch. 1.\phantom{\textcite{Dahl1971}}
					\end{enumerate}
			
				\item[4.] {\bf ¿Qu\'e es la Democracia? Cuestiones conceptuales}.
					\begin{enumerate}
					 \item Schmitter, Philippe and Terry Karl. 1991. \href{https://github.com/hbahamonde/Ciencia_Politica_I/raw/master/Readings/Schmitter_Karl.pdf}{``What Democracy Is...and Is Not.''} \emph{Journal of Democracy}, 2(3): 75---88.\phantom{\textcite{Schmitter1991}}
					 
					 \item Collier, David, and Steven Levitsky. 1997. \href{https://github.com/hbahamonde/Ciencia_Politica_I/raw/master/Readings/Collier_Levitsky.pdf}{``Democracy with Adjectives: Conceptual Innovation in Comparative Research.''} \emph{World Politics}, 49(3): 430---451.\phantom{\textcite{Collier1997}} 
					\end{enumerate}
			
				\item[5.] {\bf ¿C\'omo se mide la Democracia? Binario versus continuo}.
					\begin{enumerate}
						\item Munck, Gerardo and Jay Verkuilen. 2002. \href{https://github.com/hbahamonde/Ciencia_Politica_I/raw/master/Readings/Munck_Verkuilen.pdf}{``Conceptualizing and Measuring Democracy: Evaluating Alternative Indices.''} \emph{Comparative Political Studies}, 35(1): 5---34.\phantom{\textcite{Munck2002}}% --not binary definition 
						
						\item Mike Alvarez, Jose Antonio Cheibub, Fernando Limongi and Adam Przeworki. 1996. \href{https://github.com/hbahamonde/Ciencia_Politica_I/raw/master/Readings/Classifying_Alvarez.pdf}{``Classifying Political Regimes.''} \emph{Studies In Comparative International Development}, 31(2): 3---36.\phantom{\textcite{Alvarez1996}} % binary
					\end{enumerate}
			
				\item[6.] {\bf ¿C\'omo Nacen las Democracias? Pre-requisitos sociales versus pre-requisitos econ\'omicos}.  
					\begin{enumerate}
						\item Lipset, Seymour. 1959. \href{https://github.com/hbahamonde/Ciencia_Politica_I/raw/master/Readings/Lipset.pdf}{``Some Social Requisites of Democracy: Economic Development and Political Legitimacy.''} \emph{American Political Science Review}, 53(1): 69---105.\phantom{\textcite{Lipset1959}}

						\item Przeworski, Adam and Limongi, Fernando. 1997. \href{https://github.com/hbahamonde/Ciencia_Politica_I/raw/master/Readings/Przeworski_Limongi_1997.pdf}{``Modernization: Theories and Facts.''} \emph{World Politics}, 49(1): 155---183.\phantom{\textcite{Przeworski1997}}
          \end{enumerate}
				
				

				\item[7.] {\bf ¿C\'omo Nacen las Democracias? El rol de las \'elites econ\'omicas}. 
					\begin{enumerate}
						\item Moore, Barrington. 1966. \href{https://github.com/hbahamonde/Ciencia_Politica_I/raw/master/Readings/Moore.pdf}{\emph{Social Origins of Dictatorship and Democracy: Lord and Peasant in the Making of the Modern World}}. Penguin Books. Ch. 7 (``The Democratic Route to Modern Society'').\phantom{\textcite{Moore:1966tn}}
						
						\item Collier, Ruth. 1999. \href{https://github.com/hbahamonde/Ciencia_Politica_I/raw/master/Readings/Collier.pdf}{\emph{Paths toward Democracy: The Working Class and Elites in Western Europe and South America}}. Cambridge University Press. Ch. 1 (``Introduction: Elite Conquest or Working-Class Triumph'').\phantom{\textcite{Collier:1999rz}}
					\end{enumerate}

				\item[8.] {\bf ¿C\'omo Nacen las Democracias? Democracia y redistribuci\'on econ\'omica (I)}.  
					\begin{enumerate}
						\item Boix, Carles. 2003. \href{https://github.com/hbahamonde/Ciencia_Politica_I/raw/master/Readings/Boix_2003.pdf}{\emph{Democracy and Redistribution}}. Cambridge University Press. Ch. 1.\phantom{\textcite{Boix:2003db}}

						\item Acemoglu, Daron and Robinson, James. 2009. \href{https://github.com/hbahamonde/Ciencia_Politica_I/raw/master/Readings/Acemoglu_Robinson.pdf}{\emph{Economic Origins of Dictatorship and Democracy}}. Cambridge University Press. Ch. 2.1---2.6.\phantom{\textcite{Acemoglu:1996rm}}
					\end{enumerate}


				\item[9.] {\bf ¿C\'omo Nacen las Democracias? Democracia y redistribuci\'on econ\'omica (II)}.  
					\begin{enumerate}
						\item Haggard, Stephan and Kaufman, Robert. 2012. \href{https://github.com/hbahamonde/Ciencia_Politica_I/raw/master/Readings/Haggard_Kaufman.pdf}{``Inequality and Regime Change: Democratic Transitions and the Stability of Democratic Rule.''} \emph{American Political Science Review}, 106(03): 495---516.\phantom{\textcite{Haggard2012}}

						\item Ansell, Ben and Samuels, David. 2014. \href{https://github.com/hbahamonde/Ciencia_Politica_I/raw/master/Readings/Ansell_Samuels.pdf}{\emph{Inequality and Democratization: An Elite-Competition Approach}}. Cambridge University Press. Ch. 1---2.\phantom{\textcite{Ansell:2014ty}}
					\end{enumerate}

				\item[10.] {\bf Partidos Pol\'iticos: formaci\'on y competencia ideol\'ogica}.
					\begin{enumerate}
						\item Aldrich, John. 1995. \href{https://github.com/hbahamonde/Ciencia_Politica_I/raw/master/Readings/Aldrich_1995.pdf}{\emph{Why Parties? The Origins and Transformation of Political Parties in America}}. The University of Chicago Press. Ch. 1.\phantom{\textcite{Aldrich:1995td}}
						
						\item Downs, Anthony. 1957. \href{https://github.com/hbahamonde/Ciencia_Politica_I/raw/master/Readings/Downs.pdf}{\emph{An Economic Theory of Democracy}}. Harper and Row. Ch. 2---3.\phantom{\textcite{Downs:1957vg}}
					\end{enumerate}
			
				\item[11.] {\bf Clientelismo: ¿una falla democr\'atica?} 
					\begin{enumerate}
						\item Kitschelt, Herbert. 2000. \href{https://github.com/hbahamonde/Ciencia_Politica_I/raw/master/Readings/Kitschelt.pdf}{``Linkages between Citizens and Politicians in Democratic Polities.''} \emph{Comparative Political Studies}, 33(6-7): 845---879.\phantom{\textcite{Kitschelt2000}}
					
						\item Auyero, Javier. 2000. \href{https://github.com/hbahamonde/Ciencia_Politica_I/raw/master/Readings/Auyero.pdf}{``The Logic of Clientelism in Argentina: An Ethnographic Account.''} \emph{Latin American Research Review}, 35(3): 55---81.\phantom{\textcite{Auyero2000}}
					\end{enumerate}

				\item[11.] {\bf Clientelismo: avances metodol\'ogicos.} 
					\begin{enumerate}
						\item González-Ocantos, Ezequiel, Chad de Jonge, Carlos Meléndez, Javier Osorio, and David Nickerson. 2012. \href{https://github.com/hbahamonde/Ciencia_Politica_I/raw/master/Readings/Ocantos.pdf}{Vote Buying and Social Desirability Bias: Experimental Evidence from Nicaragua}. American Journal of Political Science 56 (1): 202–217.
					
						\item Bahamonde, Hector. 2020. \href{https://github.com/hbahamonde/Ciencia_Politica_I/raw/master/Readings/Bahamonde.pdf}{``Still for Sale: The Micro-Dynamics of Vote Selling in the United States, Evidence from a List Experiment.''} \emph{Acta Politica}, forthcoming.\phantom{\textcite{Bahamonde2020a}}
					\end{enumerate}
				
				\item[12.] {\bf Sistemas Electorales: or\'igenes y consecuencias}.
					\begin{enumerate}
						\item Lijphart, Arend. 1990. \href{https://github.com/hbahamonde/Ciencia_Politica_I/raw/master/Readings/Lijphart_1990.pdf}{``The Political Consequences of Electoral Laws.''} \emph{The American Political Science Review}, 84(2): 481---496.\phantom{\textcite{Lijphart1990}}

						\item Boix, Carles. 1999. \href{https://github.com/hbahamonde/Ciencia_Politica_I/raw/master/Readings/Boix.pdf}{``Setting the Rules of the Game: The Choice of Electoral Systems in Advanced Democracies.''} \emph{The American Political Science Review}, 93(3): 609---624.\phantom{\textcite{Boix:1999tj}} 
									
						\item[{\Pointinghand}] {\bf El caso Chileno (papers sugeridos):}
									\begin{enumerate}
										\item Navia, Patricio. 2005. \href{https://github.com/hbahamonde/Ciencia_Politica_I/raw/master/Readings/Navia_2005.pdf}{``La transformaci\'on de votos en esca\~nos: leyes electorales en Chile, 1833-2004.''} \emph{Pol\'itica y Gobierno}, 12(2): 233---276.\phantom{\textcite{Navia2005}}
										\item Pastor, Daniel. 2004. \href{https://github.com/hbahamonde/Ciencia_Politica_I/raw/master/Readings/Pastor_2004.pdf}{``Origins of the Chilean Binominal Election System.''} \emph{Revista de Ciencia Pol\'itica}, 24(1): 38---57.\phantom{\textcite{Pastor2004}}
										\item El actual sistema electoral Chileno [\href{https://www.servel.cl/nuevo-sistema-electoral-chileno-metodo-dhont-2/}{Link Servel}].
									\end{enumerate}
					\end{enumerate}

				
				\item[13.] {\bf Parlamentarismo y Presidencialismo: ventajas y desventajas de cada uno}.
					\begin{enumerate}
						\item Linz, Juan. 1994. \href{https://github.com/hbahamonde/Ciencia_Politica_I/raw/master/Readings/Linz.pdf}{\emph{Presidential or Parliamentary Democracy: Does it Make a Difference?}} In ``The Failure of Presidential Democracy: The Case of Latin America,'' Juan Linz and Arturo Valenzuela (Eds.), pp. 1---18. Johns Hopkins University Press.\phantom{\textcite{Linz1985}}% Parla
						
						\item Mainwaring, Scott and Shugart, Matthew. 1997. \href{https://github.com/hbahamonde/Ciencia_Politica_I/raw/master/Readings/Mainwaring_Shugart.pdf}{``Juan Linz, Presidentialism, and Democracy: A Critical Appraisal.''} \emph{Comparative Politics}, 29(4): 449---471.\phantom{\textcite{Mainwaring1985}}% Presiden
					\end{enumerate}


        \item[14.] {\bf ¿C\'omo Mueren las Democracias?}
          \begin{enumerate}
            \item O'Donnell, Guillermo. 1973. \href{https://github.com/hbahamonde/Ciencia_Politica_I/raw/master/Readings/ODonnell.pdf}{``Modernization and Bureaucratic-Authoritarianism: Studies in South American Politics''}. Univ of California Intl. {\color{red}pp?} \phantom{\textcite{ODonnell1973}}
            \item Steven Levitsky and Daniel Ziblatt. 2018. \href{https://github.com/hbahamonde/Ciencia_Politica_I/raw/master/Readings/Como_Mueren_Democracias.pdf}{``C\'omo Mueren las Democracias''}. Ariel. {\color{red}pp?}\phantom{\textcite{Levitsky:2018aa}}
          \end{enumerate}
  

  \end{enumerate}




{\huge{\color{blue}\Pointinghand}} {\large{\color{red}{\bf Midterm para la casa. Entra todo lo visto hasta el momento}.}}

		\item {\bf \underline{R\'egimenes No Democr\'aticos}}
			\begin{enumerate}
				
				\item[15.] {\bf ¿Qu\'e es el Totalitarismo?}
					\begin{enumerate}
						\item Linz, Juan. 2000. \href{https://github.com/hbahamonde/Ciencia_Politica_I/raw/master/Readings/Linz_2000.pdf}{\emph{Totalitarian and Authoritarian Regimes}}. Lynne Rienner Publishers. Ch. 2.\phantom{\textcite{Linz2000}}
					\end{enumerate}

				\item[14.] {\bf ¿Qu\'e es el Autoritarismo?}
					\begin{enumerate}
						\item Linz, Juan. 2000. \href{https://github.com/hbahamonde/Ciencia_Politica_I/raw/master/Readings/Linz_2000.pdf}{\emph{Totalitarian and Authoritarian Regimes}}. Lynne Rienner Publishers. Ch. 4.\phantom{\textcite{Linz2000}}
					\end{enumerate}


        \item[15.] {\bf R\'egimenes no-democr\'aticos ``mixtos'': Los R\'egimenes Personalistas y ``Tradicionales''}.
          \begin{enumerate}
            \item Linz, Juan. 2000. \href{https://github.com/hbahamonde/Ciencia_Politica_I/raw/master/Readings/Linz_2000.pdf}{\emph{Totalitarian and Authoritarian Regimes}}. Lynne Rienner Publishers. Ch. 3.\phantom{\textcite{Linz2000}}
          \end{enumerate}


				\item[16.] {\bf Supervivencia de los L\'ideres No Democr\'aticos}.
					\begin{enumerate}
						\item Gandhi, Jennifer and Przeworski, Adam. 2007. \href{https://github.com/hbahamonde/Ciencia_Politica_I/raw/master/Readings/Gandhi_Przeworski.pdf}{``Authoritarian Institutions and the Survival of Autocrats.''} \emph{Comparative Political Studies}, 40(11): 1279---1301.\phantom{\textcite{Gandhi:2007et}}
						
						\item Magaloni, Beatriz. 2008. \href{https://github.com/hbahamonde/Ciencia_Politica_I/raw/master/Readings/Magaloni.pdf}{``Credible Power-Sharing and the Longevity of Authoritarian Rule.''} \emph{Comparative Political Studies}, 41(4-5): 715---741.\phantom{\textcite{Magaloni2008}}
					\end{enumerate}
				


				\item[17.] {\bf Elecciones en Contextos No Democr\'aticos: ``Autoritarismos Competitivos''}.
					\begin{enumerate}
						\item Levitsky, Steven and Way, Lucan. 2002. \href{https://github.com/hbahamonde/Ciencia_Politica_I/raw/master/Readings/Levitsky_Way.pdf}{``Elections Without Democracy: The Rise of Competitive Authoritarianism.''} \emph{Journal of Democracy}, 13(2): 51---65.\phantom{\textcite{Levitsky2002}}
						\item Carothers, Christopher. 2018. \href{https://github.com/hbahamonde/Ciencia_Politica_I/raw/master/Readings/Carothers.pdf}{``The Surprising Instability of Competitive Authoritarianism.''} \emph{Journal of Democracy}, 29(4): 129---135.\phantom{\textcite{Carothers2018}}
					\end{enumerate}




				\item[18.] {\bf Pero ¿es importante el tipo de r\'egimen?}
					\begin{enumerate}
						\item Przeworski, Adam and Limongi, Fernando. 1993. \href{https://github.com/hbahamonde/Ciencia_Politica_I/raw/master/Readings/Przeworski_Limongi_1993.pdf}{``Political Regimes and Economic Growth.''} \emph{Journal of Economic Perspectives}, 7(3): 51---69.\phantom{\textcite{Przeworski1993}} 
						
						\item Ross, Michael. 2006. \href{https://github.com/hbahamonde/Ciencia_Politica_I/raw/master/Readings/Ross.pdf}{``Is Democracy Good for the Poor?''} \emph{American Journal of Political Science}, 50(4): 860---874.\phantom{\textcite{Ross2006}}
					\end{enumerate}

				
				\item[19.] {\bf Redistribici\'on Econ\'omica en Democracia y en Dictadura} 
					\begin{enumerate}
						\item Albertus, Michael. 2015. \href{https://github.com/hbahamonde/Ciencia_Politica_I/raw/master/Readings/Albertus.pdf}{\emph{Autocracy and Redistribution. The Politics of Land Reform}}. Cambridge University Press. Ch. 3.\phantom{\textcite{Albertus:2015aa}}
						\item Bahamonde, Hector and Mart Trasberg. 2021. \href{https://github.com/hbahamonde/Ciencia_Politica_I/raw/master/Readings/Bahamonde_Trasberg.pdf}{\emph{Inclusive Institutions, Unequal Outcomes: Democracy, State Capacity and Income Inequality}}. European Journal of Political Economy (under review).\phantom{\textcite{Bahamonde2021}}
					\end{enumerate}

			\end{enumerate}



	\item {\bf \underline{Estado}}
		\begin{enumerate}
			
			\item[20.] {\bf Formaci\'on del estado: El estado europeo a la luz de la teor\'ia econ\'omica y sociol\'ogica}.
				\begin{enumerate} 
					\item Olson, Mancur. 1993. \href{https://github.com/hbahamonde/Ciencia_Politica_I/raw/master/Readings/Olson.pdf}{``Dictatorship, Democracy, and Development.''} \emph{The American Political Science Review}, 87(3): 567---576.\phantom{\textcite{Olson1993}}
					
					\item Tilly, Charles. 1985. \href{https://github.com/hbahamonde/Ciencia_Politica_I/raw/master/Readings/Tilly.pdf}{\emph{War Making as Organized Crime}}. In ``Bringing the State Back In,'' Peter Evans, Dieter Rueschemeyer and Theda Skocpol (eds.). New York: Cambridge University Press, pp. 169---187.\phantom{\textcite{Tilly1985}}
				\end{enumerate}
			
			\item[21.] {\bf Formaci\'on del estado: El estado europeo a la luz de la teor\'ia de la sociololog\'ia del derecho}.
				\begin{enumerate} 
					\item Strayer, Joseph. 1973. \href{https://github.com/hbahamonde/Ciencia_Politica_I/raw/master/Readings/Strayer.pdf}{\emph{On the Medieval Origins of the Modern State}}. Princeton University Press. Ch. 1.\phantom{\textcite{Strayer:2005uq}}
				\end{enumerate}

			\item[22.] {\bf Formaci\'on del estado: Latinoam\'erica y Sudeste Asi\'atico}.
				\begin{enumerate}
					\item Centeno, Miguel Angel. 1997. \href{https://github.com/hbahamonde/Ciencia_Politica_I/raw/master/Readings/Centeno.pdf}{``Blood and Debt: War and Taxation in Nineteenth-Century Latin America.''} \emph{American Journal of Sociology}, 102(6): 1565---1605.\phantom{\textcite{Centeno1997}} 
					
					\item Slater, Dan. 2008. \href{https://github.com/hbahamonde/Ciencia_Politica_I/raw/master/Readings/Slater.pdf}{``Can Leviathan be Democratic? Competitive Elections, Robust Mass Politics, and State Infrastructural Power.''} \emph{Studies in Comparative International Development}, 43(3-4): 252---272.\phantom{\textcite{Slater2008}}
				\end{enumerate}


      \item[23.] {\bf Or\'igenes coloniales y la econom\'ia pol\'itica del estado Latinoam\'ericano}.
        \begin{enumerate}
          \item Matthew Lange, James Mahoney and Matthias vom Hau. 2006. \href{https://github.com/hbahamonde/Ciencia_Politica_I/raw/master/Readings/Mahoney_et_al_2006.pdf}{``Colonialism and Development: A Comparative Analysis of Spanish and British Colonies.''} \emph{American Journal of Sociology}, 111(5): 1412---1462.\phantom{\textcite{Lange2006}}
          \item Haber, Stephen. 1991. \href{https://github.com/hbahamonde/Ciencia_Politica_I/raw/master/Readings/Haber_1991.pdf}{``Industrial Concentration and the Capital Markets: A Comparative Study of Brazil, Mexico, and the United States, 1830–1930.''} \emph{The Journal of Economic History}, 51(3): 559---580.\phantom{\textcite{Haber1991}}
        \end{enumerate}

			\item[24.] {\bf ¿Son los Booms Econ\'omicos Positivos para la Formaci\'on del Estado?: \'Africa y Latinoam\'erica}.
				\begin{enumerate}
					\item Saylor, Ryan. 2014. \href{https://github.com/hbahamonde/Ciencia_Politica_I/raw/master/Readings/Saylor.pdf}{\emph{State Building in Boom Times: Commodities and Coalitions in Latin America and Africa}}. Oxford University Press. Ch. 1---p. 39.\phantom{\textcite{Saylor:2014aa}}
					
					\item Ross, Michael. 2012. \href{https://github.com/hbahamonde/Ciencia_Politica_I/raw/master/Readings/Ross_2012.epub}{\emph{The Oil Curse: How Petroleum Wealth Shapes the Development of Nations}}. Princeton University Press. Ch. 3---p. 93.\phantom{\textcite{Ross:2012nr}}
				\end{enumerate}
		

		\end{enumerate}

{\huge{\color{blue}\Pointinghand}} {\large{\color{red}{\bf Entrega del temario para el ensayo final (para la casa). Entra todo lo visto hasta el momento. Entrega en dos semanas m\'as en \texttt{uCampus}.}}}

\end{enumerate}



			


			



\newpage
\pagenumbering{roman}
\setcounter{page}{1}
\printbibliography



\end{document}


