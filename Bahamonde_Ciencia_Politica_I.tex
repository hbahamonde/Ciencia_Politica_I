%LaTeX Curriculum Vitae Template
%
% Copyright (C) 2004-2009 Jason Blevins <jrblevin@sdf.lonestar.org>
% http://jblevins.org/projects/cv-template/
%
% You may use use this document as a template to create your own CV
% and you may redistribute the source code freely. No attribution is
% required in any resulting documents. I do ask that you please leave
% this notice and the above URL in the source code if you choose to
% redistribute this file.

\documentclass[letterpaper]{article}

\usepackage{hyperref}
\hypersetup{
    bookmarks=true,         % show bookmarks bar?
    unicode=false,          % non-Latin characters in Acrobat’s bookmarks
    pdftoolbar=true,        % show Acrobat’s toolbar?
    pdfmenubar=true,        % show Acrobat’s menu?
    pdffitwindow=true,     % window fit to page when opened
    pdfstartview={FitH},    % fits the width of the page to the window
    pdftitle={My title},    % title
    pdfauthor={Author},     % author
    pdfsubject={Subject},   % subject of the document
    pdfcreator={Creator},   % creator of the document
    pdfproducer={Producer}, % producer of the document
    pdfkeywords={keyword1} {key2} {key3}, % list of keywords
    pdfnewwindow=true,      % links in new window
    colorlinks=true,       % false: boxed links; true: colored links
    linkcolor=blue,          % color of internal links (change box color with linkbordercolor)
    citecolor=blue,        % color of links to bibliography
    filecolor=blue,      % color of file links
    urlcolor=blue           % color of external links
}



\usepackage{geometry}
\usepackage{import} % To import email.
\usepackage{marvosym} % face package
\usepackage{xcolor,color}
\usepackage{fontawesome}
\usepackage{amssymb} % for bigstar
\usepackage{epigraph}

% Comment the following lines to use the default Computer Modern font
% instead of the Palatino font provided by the mathpazo package.
% Remove the 'osf' bit if you don't like the old style figures.
\usepackage[T1]{fontenc}
\usepackage[sc,osf]{mathpazo}

% Set your name here
\def\name{Ciencia Pol\'itica I - AP2107}

% Replace this with a link to your CV if you like, or set it empty
% (as in \def\footerlink{}) to remove the link in the footer:
\def\footerlink{}
% \href{http://www.hectorbahamonde.com}{www.HectorBahamonde.com}

% The following metadata will show up in the PDF properties
\hypersetup{
  colorlinks = true,
  urlcolor = blue,
  pdfauthor = {\name},
  pdfkeywords = {intro to social sciences},
  pdftitle = {\name: Syllabus},
  pdfsubject = {Syllabus},
  pdfpagemode = UseNone
}

\geometry{
  body={6.5in, 8.5in},
  left=1.0in,
  top=1.25in
}

% Customize page headers
\pagestyle{myheadings}
\markright{{\tiny \name}}
\thispagestyle{empty}

% Custom section fonts
\usepackage{sectsty}
\sectionfont{\rmfamily\mdseries\Large}
\subsectionfont{\rmfamily\mdseries\itshape\large}

% Don't indent paragraphs.
\setlength\parindent{0em}

% Make lists without bullets
\renewenvironment{itemize}{
  \begin{list}{}{
    \setlength{\leftmargin}{1.5em}
  }
}{
  \end{list}
}


% email input begin
\newread\fid
\newcommand{\readfile}[1]% #1 = filename
{\bgroup
  \endlinechar=-1
  \openin\fid=#1
  \read\fid to\filetext
  \loop\ifx\empty\filetext\relax% skip over comments
    \read\fid to\filetext
  \repeat
  \closein\fid
  \global\let\filetext=\filetext
\egroup}
\readfile{/Users/hectorbahamonde/RU/Bibliografia_PoliSci/email.txt}
% email input end


%%% bib begin
\usepackage[american]{babel}
\usepackage{csquotes}
%\usepackage[style=chicago-authordate,doi=false,isbn=false,url=false,eprint=false]{biblatex}

\usepackage[authordate,isbn=false,doi=false,url=false,eprint=false]{biblatex-chicago}
\DeclareFieldFormat[article]{title}{\mkbibquote{#1}} % make article titles in quotes
\DeclareFieldFormat[thesis]{title}{\mkbibemph{#1}} % make theses italics

\AtEveryBibitem{\clearfield{month}}
\AtEveryCitekey{\clearfield{month}}

\addbibresource{/Users/hectorbahamonde/RU/Bibliografia_PoliSci/library.bib} 
\addbibresource{/Users/hectorbahamonde/RU/Bibliografia_PoliSci/Bahamonde_BibTex2013.bib} 

% USAGES
%% use \textcite to cite normal
%% \parencite to cite in parentheses
%% \footcite to cite in footnote
%% the default can be modified in autocite=FOO, footnote, for ex. 
%%% bib end




\begin{document}

% Place name at left
%{\huge \name}

% Alternatively, print name centered and bold:
\centerline{\huge \bf \name}

\epigraph{\emph{Statistics: ``science dealing with data about the condition of a state or community''}}{Gottfried Aschenwall, 1770}


\vspace{0.25in}

\begin{minipage}{0.45\linewidth}
 Universidad de O$'$Higgins \\
  Instituto de Ciencias Sociales \\
  Rancagua, Chile\\
  \\
  \\

\end{minipage}
\hspace{4cm}\begin{minipage}{0.45\linewidth}
  \begin{tabular}{ll}
{\bf \'Ultima actualizaci\'on}: \today. \\
 {\bf Descarga la \'ultima versi\'on} \href{https://github.com/hbahamonde/Ciencia_Politica_I/raw/master/Bahamonde_Ciencia_Politica_I.pdf}{aqu\'i}.%\\
   %{\bf {\color{red}{\scriptsize Not intended as a definitive version}}} %\\
    \\
    \\
    \\
    \\
    \\
  \end{tabular}
\end{minipage}



\subsection*{Aspectos Log\'isticos}


\vspace{1mm}
{\bf Profesor}: H\'ector Bahamonde, PhD.\\
\texttt{e:}\href{mailto:hector.bahamonde@uoh.cl}{\texttt{hector.bahamonde@uoh.cl}}\\
\texttt{w:}\href{http://www.hectorbahamonde.com}{\texttt{www.hectorbahamonde.com}}\\
{\bf Office Hours}: Toma una hora \href{https://calendly.com/bahamonde/officehours}{\texttt{aqu\'i}}.


\vspace{5mm}
{\bf Hora de c\'atedra}: {\input{/Users/hectorbahamonde/RU/Teaching/Ciencia_Politica_I/time_class_1.txt}\unskip}; {\input{/Users/hectorbahamonde/RU/Teaching/Ciencia_Politica_I/time_class_2.txt}\unskip}.\\
{\bf Lugar de c\'atedra}: A307.\\
{\bf Acceso a materiales del curso}: \href{https://ucampus.uoh.cl/uoh/2019/2/AP2107/1/}{\texttt{uCampus}}.

\vspace{5mm}
{\bf Ayudante de c\'atedra (TA)}: {\input{/Users/hectorbahamonde/RU/Teaching/Ciencia_Politica_I/ta_name.txt}\unskip}.\\
\texttt{e:}\href{mailto:gonzalo.barria@uoh.cl}{\texttt{gonzalo.barria@uoh.cl}}\\
{\bf TA Bio}: Gonzalo Barr\'ia es Cientista Pol\'itico (PUC) y Mag\'ister en Ciencia Pol\'itica (PUC).\\
{\bf Hora de ayudant\'ia}: 10.15-11.45.\\
{\bf Lugar de ayudant\'ia}: A307.\\

\vspace{5mm}
{\bf Carrera}:	Administraci\'on P\'ublica.\\
{\bf Eje de Formaci\'on}: L\'inea Ciencia Pol\'itica.\\
{\bf Semestre/A\~no}:	Quinto Semestre/2020.\\
{\bf Pre-requisitos}: M\'etodos de Investigaci\'on.\\
{\bf SCT}: 6.\\
{\bf Horas semanales}: C\'atedra (3 horas), Ayudant\'ia	(1.5 horas).\\
{\bf Semanas}:	15.\\



\subsection*{Motivaci\'on: ¿Por qu\'e tomar este curso?}

\emph{¿Qu\'e efecto tiene la educaci\'on sobre los ingresos? ¿C\'omo podemos evaluar los efectos de una reforma educacional? ¿La legalizaci\'on de la mariguana aumenta su consumo? ¿Qu\'e candidato/a ganar\'ia la elecci\'on presidencial si \'esta fuera ma\~nana?} 
\\
\\
Las entidades p\'ublicas gu\'ian sus decisiones estrat\'egicas en base informaci\'on, i.e. datos. Esto ha tomado incluso m\'as importancia en la actualidad, donde ha habido una digitalizaci\'on de los datos sociales. Es fundamental que los cientistas sociales, y en particular, los/las administradores/as p\'ublicos, sepan c\'omo usar estos datos. A\'un m\'as, el quehacer social en general, est\'a constantemente produciendo datos. Cada vez que usas \emph{Twitter}, pides un \emph{Uber}, env\'ias un e-mail, votas, respondes una encuesta, est\'as produciendo datos sociales. Piensa en lo siguiente: si bien es cierto que hace unos diez a\~nos atr\'as \emph{faltaban} datos, hoy en d\'ia los datos \emph{sobran}. El desaf\'io actual consiste en saber c\'omo analizarlos correctamente, y as\'i ayudar a los tomadores de decisiones. Esto es importante. {\bf Ma\~nana tu podr\'ias ser un/a analista en una de las decenas de Departamentos de Estudios repartidas en la administraci\'on del Estado}. {\bf Este curso te prepara para ese mundo} (incluyendo el mundo de la consultor\'ia).
\\
\\
Aunque lo que aprenderemos es altamente num\'erico y matem\'atico, no te confundas. Estos m\'etodos no son infalibles, y no nos contar\'an ``la verdad'' (si es que algo as\'i existiera). A\'un necesitas ser muy critico(a). Como ver\'as, {\bf la \emph{estad\'istica inferencial} (que es el objeto de este curso) es un \emph{arte}, no una \emph{ciencia}}. Los n\'umeros nos sugerir\'an ciertas ideas, pero aun as\'i nuestro trabajo ser\'a \emph{interpretar} estos resultados. No seas obediente. Se cr\'itico/a, auto-cr\'itico/a. Sospecha de tus propios resultados y el de los dem\'as. Mal que mal, estaremos haciendo {\bf inferencias} (no \emph{certezas}) estad\'isticas. Como veremos, el fantasma de este semestre se llamar\'a \emph{incertidumbre}. 
\\
\\
Honestamente, espero que este curso cautive tu atenci\'on, y simiente tu curiosidad intelectual, sobre todo, mostr\'andote que nuestro objeto de estudio (la sociedad) es apasionante. 
\\
\\
\emph{Bienvenid$@$s!}


\subsection*{Prop\'osito Formativo}

Los problemas p\'ublicos han sido entendidos y definidos como resultado de los principales debates pol\'iticos y filos\'oficos a lo largo de la historia de Occidente. Este curso pondr\'a \'enfasis en que los/las estudiantes puedan reconocer aquellos aportes e ideas que mayor impacto han tenido en el debate politol\'ogico contempor\'aneo, con el objetivo que comprendan, analicen e interpreten las principales ideas que han estado presentes en el debate te\'orico pol\'itico respecto a los significados subyacentes a la acci\'on pol\'itica y los apliquen en contextos propios de la gesti\'on p\'ublica.

\subsection*{Objetivos Generales del Curso}

El gran objetivo de este curso, es poder 

Este curso est\'a dividido en tres grandes unidades. Cada unidad tiene sus diferentes evaluaciones.


\begin{enumerate}
	\item 1
	\item 2
	\item 3
\end{enumerate}
 

\subsection*{Objetivos Espec\'ificos del Curso}

\begin{itemize}
	\item[$\bullet$] 1
	\item[$\bullet$] 2
	\item[$\bullet$] 3
\end{itemize}


\begin{itemize}
		\item[{\color{red}\Pointinghand}] Se espera que los estudiantes hagan sus respectivas lecturas \emph{antes} de cada clase para poder participar en el debate cr\'itico que haremos en cada una de ellas. Tambi\'en se espera que los/las estudiantes hagan los ejercicios pr\'acticos clase a clase.
\end{itemize}



\subsection*{Competencias Transversales}


\begin{itemize}
	\item[$\bullet$] Utiliza y aplica un pensamiento hol\'istico, cr\'itico, l\'ogico y creativo para comprender y explicar los fen\'omenos propios de su entorno.
	\item[$\bullet$] Desarrolla su labor con apego al Estado de Derecho y la institucionalidad democr\'atica, guiado por los principios de transparencia, imparcialidad, eficacia, eficiencia, probidad, responsabilidad. 
	\item[$\bullet$] Incorpora la tecnolog\'ia y aplica t\'ecnicas y herramientas apropiadas para la comprensi\'on, an\'alisis y resoluci\'on de problemas p\'ublicos.
\end{itemize}


\subsection*{Integridad Acad\'emica}

En este curso {\bf el plagio y la copia est\'an absolutamente prohibidas}. Nuestra casa de estudios aun no cuenta con un protocolo de integridad acad\'emica. Mientras tanto, seguiremos las reglas de conducta de  \href{www.extension.harvard.edu/resources-policies/student-conduct/academic-integrity}{\texttt{Harvard University}}. Tod$@$s l$@$s estudiantes de este curso deber\'an familiarizarse con estas reglas y definiciones desde la primera clase. Aunque de manera no exhaustiva, el plagio se define como \emph{hacer pasar por propias las ideas de otros}. De igual manera, la copia se define como \emph{copiar las ideas de otros}. {\bf La evidencia de cualquier falta a la integridad acad\'emica ser\'a sancionada con un 1 en todo el curso}. No habr\'an excepciones. 
\\
\\
Aunque tu ser\'as \emph{absolutamente} responsable de c\'omo usar referencias bibliogr\'aficas apropiadamente, el TA y yo cubriremos con el curso un set de buenas pr\'acticas que te ayudar\'an a no caer en un plagio por omisi\'on. Mientras tanto, puedes \href{https://www.extension.harvard.edu/resources-policies/resources/avoiding-plagiarism}{\texttt{consultar}} los siguientes consejos.



\subsection*{Etiquette: C\'atedra y Ayudant\'ia}
 

\begin{itemize}
	\item[$\bullet$] No llegues tarde. La sala de clases se cierra despu\'es de los primeros 15 minutos. Esta regla es importante, y no tendr\'a excepciones.
  \item[$\bullet$] No te retires antes. Esta regla es importante, y no tendr\'a excepciones.
	\item[$\bullet$] No comas en clases. Bebestibles, tales como caf\'e y t\'e est\'an OK.
	\item[$\bullet$] {\bf No se pueden ocupar \emph{laptops}}. No se pueden ocupar celulares ni \emph{tablets}, ni otros aparatos digitales. No habr\'an excepciones. Los celulares deber\'an estar apagados, no en silencio. Aquellos estudiantes que no respeten esta regla, ser\'an invitados a salir de la sala. %No puedes sacar fotos a las diapositivas.
	\item[$\bullet$] La asistencia es obligatoria (y parte de tu nota de participaci\'on). Si faltaste a una clase, cons\'iguete los apuntes con un compa\~nero/a. Yo no ofrezco clases particulares de mi clase. Sin embargo, si tienes preguntas espec\'ificas, \href{https://calendly.com/bahamonde/officehours}{toma una hora} conmigo. 
	\item[{\color{red}\Pointinghand}] Si vives en un lugar donde hay mala conectividad de Internet, planea tu trabajo de manera eficiente. Por ejemplo, no esperes hasta ultimo minuto para enviar tus trabajos via uCampus. {\bf Trabajos que queden fuera de plazo, tendran un 1 autom\'aticamente}.
\end{itemize}



\subsection*{Evaluaciones}

\begin{enumerate}

	% Participation
	\item {\bf Lecturas, Participaci\'on, \emph{Pop Quizzes}, y Asistencia}: {\input{/Users/hectorbahamonde/RU/Teaching/Ciencia_Politica_I/percentage_participation.txt}\unskip}\%.
	
	
	\begin{itemize}
		\item[\Pointinghand] {\bf La asistencia a cada una de las clases y a cada una de las ayudant\'ias es obligatoria}.
	\end{itemize}

	El TA y yo asumiremos durante todo el semestre que has le\'ido. Nosotros empleamos un m\'etodo de clases interactivo, pero este m\'etodo necesita de tu participaci\'on activa en clases. \emph{Asistencia no s\'olo significa que vayas a clase; tambi\'en debes participar}. Esto significa que aunque hayas ido a todas las clases, es \emph{imposible} que tengas la nota m\'axima en asistencia si es que no has participado en clases y ayudant\'ia. Es por esto que la nota es de asistencia \emph{y} participaci\'on.
\\
\\	
	Para asegurarnos de que est\'es haciendo las lecturas, habr\'an una serie de \emph{pop quizzes} (``pruebas sorpresa'') tanto en c\'atedra como ayudant\'ia. Estos controles ser\'an cortos (5-10 minutos), y apuntan a medir si leyeron; o sabes, o no sabes. Estas pruebas se aplicar\'an completamente al azar, en cualquier momento de la clase, y sin previo aviso. Si faltaste, tendr\'as un 1 en ese control. En general, las preguntas ser\'an acerca de un concepto clave, y cuya respuesta correcta ser\'a una l\'inea (o dos, como m\'aximo). Tambi\'en puede ser que la pregunta requiera algunas l\'ineas de programaci\'on.





\end{enumerate}


\underline{En resumen}:

\begin{table}[h]
\begin{tabular}{ccc}
							& \textbf{Porcentaje} & {\bf Porcentaje Acumulado} \\
							\hline
Participaci\'on, asistencia, \emph{pop-quizzes} (c\'atedra y ayudant\'ia) & 15\%       & 15\%                 \\
\hline
Evaluaci\'on pr\'actica en clases: \#1 & 5\% & 20\%                 \\
Evaluaci\'on pr\'actica en clases: \#2 & 5\% & 25\%                 \\
Evaluaci\'on pr\'actica en clases: \#3 & 5\% & 30\%                 \\
Evaluaci\'on pr\'actica en clases: \#4 & 5\% & 35\%                 \\
Evaluaci\'on pr\'actica en clases: \#5 & 5\% & 40\%  \\
\hline
Tarea pr\'actica para la casa: \#1 	 & 10\% & 50\%    \\
Tarea pr\'actica para la casa: \#2 	 & 10\% & 60\%    \\
Tarea pr\'actica para la casa: \#3 	 & 10\% & 70\%    \\
\hline
Trabajo final grupal & 20\% & 90\% \\
Presentaci\'on grupal & 10\% & 100\% \\
\hline             
\end{tabular}
\end{table}

\subsection*{Ayudant\'ia}

Cada semana te reunir\'as con tu ayudante (``TA''). Ah\'i tendr\'as otra oportunidad para ejercitar y seguir profundizando otras tem\'aticas pendientes. En esta oportunidad, tambi\'en se revisar\'an aspectos m\'as formales de las humanidades y las ciencias sociales. 

\subsection*{Calendario}



\begin{enumerate}
		\item {\bf Introducci\'on}
			\begin{enumerate}
				\item[1.] ¿Qu\'e es la ciencia pol\'itica? % Analyzing Politics, Por Ellen Grigsby
				\item[2.] El m\'etodo comparativo. % PD&S Chs. 1-2 (1-43)
			\end{enumerate}
		\item {\bf Reg\'imenes Pol\'iticos: Democracia}
			\begin{enumerate}
				\item ¿Qu\'e es la democracia? (1) % (A) Lijphart Ch. 1-4, (B) Dahl, Robert, Polyarchy: Participation and Opposition
				\item ¿Qu\'e es la democracia? (2) % (A) Schmitter, Philippe and Terry Karl. 1991. What Democracy Is…and Is not. Journal of Democracy 2(3):75-88., (B) Collier, David, and Steven Levitsky, “Democracy with Adjectives: Conceptual Innovation in Comparative Research,” World Politics 49:3 (April 1997), pp. 430-51. 
				\item ¿C\'omo se mide la democracia? % (A) Munck, Gerardo and Jay Verkuilen. 2002. Conceptualizing and Measuring Democracy: Evaluating Alternative Indices. Comparative Political Studies 35:5-34. --not binary definition (B) Przeworski [Democracy and the Market: Political and Economic Reforms in Eastern Europe and Latin America] -- binary definition.
				\item ¿Qu\'e causa la democracia? (1) % (A) Przeworksi, Adam and Fernando Limongi. 1997. Modernization: Theories and Facts. World Politics 49: 155-183. (B) Lipset, Seymour M, “Some Social Requisites of Democracy: Economic Development and Political Legitimacy, APSR, 53, 1959, pp. 69-105, (C) Boix, Carles and Susan C. Stokes, “Endogenous Democratization,” World Politics, 55 (July 2003), pp. 517-49.
				\item ¿Qu\'e causa la democracia? (2)  % (A) Acemoglu, Johnson y Robinson, (B) Barrington Moore, (C) Haggard, S. and R. Kaufman. 2012. Inequality and Regime Change. APSR, 106, 03, pp.495-516.
				\item 
				\item Democracia en sociedades plurales. % Arend Lijphart, Democracy in Plural Societies (Yale 1977), Ch. 1 (1-20) and Ch. 5 (142-176)
				\item Partidos pol\'iticos: % (A) Why Parties, Aldrich, (B) Lijphart Chs. 5 and 8
				\item Sistemas electorales: Or\'igenes y Consecuencias % (A) Carles Boix. Setting the Rules of the Game: The Choice of Electoral Systems in Advanced Democracies. The American Political Science Review, 93(3): 609-624, 1999. ISSN 1556-5068. (B) Arend Lijphart. The Political Consequences of Electoral Laws, 1945-85. The American Political Science Review, 84(2): 481-496, 1990.
				\item Clientelismo % (A) Herbert Kitschelt. Linkages between Citizens and Politicians in Democratic Polities. Compar- ative Political Studies, 33(6-7): 845-879, sep 2000. ISSN 0010-4140. (B) Javier Auyero. The Logic of Clientelism in Argentina: An Ethnographic Account. Latin American Research Review, 35(3): 55-81, 2000.
				\item Parlamentarismo y presidencialismo: % (A) Matthew Soberg Shugart and Scott Mainwaring, “Presidentialism and Democracy in Latin America: Rethinking the Terms of the Debate,” Presidentialism and Democracy in Latin America, edited by Scott Mainwaring and Mathew Soberg Shugart, Cambridge University Press 1997, pp. 12-54., (B) Shugart, Matthew S. and John M. Carey. 1992. Presidents and Assemblies: Constitutional Design and Electoral Dynamics. New York: Cambridge University Press: Chs. 2 and 3.
				\item ¿Qu\'e es el autoritarismo? % (A) Gandhi, Jennifer, and Adam Przeworski, “Authoritarian Institutions and the Survival of Autocrats,” Comparative Political Studies 40:11 (2007): 1279-1301, (B)Beatriz Magaloni. Credible Power-Sharing and the Longevity of Authoritarian Rule. Comparative Political Studies, 41(4-5): 715-741, 2008. (C) Levitsky, Steven and Lucan Way (2002), "Elections Without Democracy: The Rise of Competitive Authoritarianism", in Journal of Democracy, 13(2):51-65.
				\item ¿Y qu\'e importa? % (A) Adam Przeworski and Fernando Limongi. Political Regimes and Economic Growth. Journal of Economic Perspectives, 7(3): 51-69, 1993. (B) Michael Ross. Is Democracy Good for the Poor? American Journal of Political Science, 50(4): 860-874, 2006.
			\end{enumerate}
	\item {\bf Estado}
		\begin{enumerate}
			\item Formaci\'on del estado moderno occidental: Europa %(A) Mancur Olson. Dictatorship, Democracy, and Development. The American Political Science Review, 87(3): 567-576, 1993. (B) Tilly, Charles. 1985. War Making as Organized Crime. In Bringing the State Back In edited by Peter Evans, Dieter Rueschemeyer and Theda Skocpol. New York: Cambridge University Press.
			\item Formaci\'on del estado moderno occidental: Latinoam\'erica y Sudeste Asi\'atico % (A) Miguel Angel Centeno. Blood and Debt: War and Taxation in Nineteenth-Century Latin America. American Journal of Sociology, 102(6): 1565-1605, 1997. (B) Dan Slater. Can Leviathan be Democratic? Competitive Elections, Robust Mass Politics, and State Infrastructural Power. Studies in Comparative International Development, 43(3-4): 252- 272, 2008
		\end{enumerate}



\end{enumerate}



			


			



\newpage
\pagenumbering{roman}
\setcounter{page}{1}
\printbibliography



\end{document}


