\documentclass[onesided]{article}\usepackage[]{graphicx}\usepackage[]{color}
% maxwidth is the original width if it is less than linewidth
% otherwise use linewidth (to make sure the graphics do not exceed the margin)
\makeatletter
\def\maxwidth{ %
  \ifdim\Gin@nat@width>\linewidth
    \linewidth
  \else
    \Gin@nat@width
  \fi
}
\makeatother

\definecolor{fgcolor}{rgb}{0.345, 0.345, 0.345}
\newcommand{\hlnum}[1]{\textcolor[rgb]{0.686,0.059,0.569}{#1}}%
\newcommand{\hlstr}[1]{\textcolor[rgb]{0.192,0.494,0.8}{#1}}%
\newcommand{\hlcom}[1]{\textcolor[rgb]{0.678,0.584,0.686}{\textit{#1}}}%
\newcommand{\hlopt}[1]{\textcolor[rgb]{0,0,0}{#1}}%
\newcommand{\hlstd}[1]{\textcolor[rgb]{0.345,0.345,0.345}{#1}}%
\newcommand{\hlkwa}[1]{\textcolor[rgb]{0.161,0.373,0.58}{\textbf{#1}}}%
\newcommand{\hlkwb}[1]{\textcolor[rgb]{0.69,0.353,0.396}{#1}}%
\newcommand{\hlkwc}[1]{\textcolor[rgb]{0.333,0.667,0.333}{#1}}%
\newcommand{\hlkwd}[1]{\textcolor[rgb]{0.737,0.353,0.396}{\textbf{#1}}}%
\let\hlipl\hlkwb

\usepackage{framed}
\makeatletter
\newenvironment{kframe}{%
 \def\at@end@of@kframe{}%
 \ifinner\ifhmode%
  \def\at@end@of@kframe{\end{minipage}}%
  \begin{minipage}{\columnwidth}%
 \fi\fi%
 \def\FrameCommand##1{\hskip\@totalleftmargin \hskip-\fboxsep
 \colorbox{shadecolor}{##1}\hskip-\fboxsep
     % There is no \\@totalrightmargin, so:
     \hskip-\linewidth \hskip-\@totalleftmargin \hskip\columnwidth}%
 \MakeFramed {\advance\hsize-\width
   \@totalleftmargin\z@ \linewidth\hsize
   \@setminipage}}%
 {\par\unskip\endMakeFramed%
 \at@end@of@kframe}
\makeatother

\definecolor{shadecolor}{rgb}{.97, .97, .97}
\definecolor{messagecolor}{rgb}{0, 0, 0}
\definecolor{warningcolor}{rgb}{1, 0, 1}
\definecolor{errorcolor}{rgb}{1, 0, 0}
\newenvironment{knitrout}{}{} % an empty environment to be redefined in TeX

\usepackage{alltt}
\usepackage{import}
\usepackage{soul}
\subimport{/Users/hectorbahamonde/RU/Bibliografia_PoliSci/}{LaTeX_Paper_Packages_And_Style}

% this below is to have endnotes at the end of the document.
\let\footnote=\endnote 

\title{TItle} % Article title



\date{\today}

%----------------------------------------------------------------------------------------

% BEGIN: add dollar sygn text mode
\lstset{
  mathescape = true,
  basicstyle = \ttfamily}
\newcommand{\dollar}{\mbox{\textdollar}}
% END: add dollar sygn text mode
\IfFileExists{upquote.sty}{\usepackage{upquote}}{}
\begin{document}
\pagenumbering{gobble} 

%\maketitle % Insert title

%\thispagestyle{fancy} % All pages have headers and footers


%%%%%%%%%%%%%%%%%%%%%%%%%%%%%%%%%%%%%%%%%%%%%%
% CONTENT
%%%%%%%%%%%%%%%%%%%%%%%%%%%%%%%%%%%%%%%%%%%%%%






\clearpage
\newpage
\pagenumbering{arabic}
\setcounter{page}{1}

\section{Midterm}

{\bf Profesor}: H\'ector Bahamonde.\\
\texttt{e:}\href{mailto:hector.bahamonde@uoh.cl}{\texttt{hector.bahamonde@uoh.cl}}\\
\texttt{w:}\href{http://www.hectorbahamonde.com}{\texttt{www.hectorbahamonde.com}}\\
{\bf Curso}: Ciencia Pol\'itica I.\\
\hspace{-5mm}{\bf Ayudante}: Gonzalo Barr\'ia.

\section*{Instrucciones}
Este es el temario para el ensayo \emph{midterm}. De las tres preguntas, debes escoger una. Debes escribir tu ensayo con una extensi\'on m\'inima de 1500 palabras (aproximadamente 3 p\'aginas) y una extensi\'on m\'axima de 2000 palabras (aproximadamente 4 p\'aginas). Tu respuesta debe tener formato ensayo: debe ser una an\'alisis cr\'itico de las lecturas. Si te limitas a resumir el texto, tendr\'as una nota deficiente. La cantidad de palabras incluye las citas, pero excluye el t\'itulo y la bibliograf\'ia. {\bf Deber\'as subir tu respuesta a uCampus no antes de las 5 pm (hora de Chile) del martes 11 de agosto de 2020 en la secci\'on \emph{Tareas}}. Tanto el ayudante como el profesor estar\'an disponibles durante horas y d\'ias laborales para responder preguntas. {\bf Los trabajos atrasados tendr\'an la nota m\'inima, sin excepci\'on---si tienes mala conexi\'on de Internet, planifica tu trabajo con mucha anticipaci\'on, y sube tu trabajo lo antes posible}. Revisa tu trabajo al menos dos veces, y preoc\'upate de citar adecuadamente: {\bf el plagio significa nota m\'inima}. \ul{Tu respuesta debe citar cada trabajo nombrado en la pregunta una vez como m\'aximo \emph{y} como m\'inimo (con una extensi\'on m\'axima de una oraci\'on de dos l\'ineas por cita)}. El \'unico tipo de archivo aceptado es un archivo de texto con extensi\'on \texttt{doc} o \texttt{docx}.

Escoge una pregunta:

\section*{Temario}

\begin{enumerate}
	\item Pensando en la literatura de democracia, espec\'ificamente en \textcite{Lijphart2012,Schmitter1991,Collier1997}, def\'inela (1 p\'agina m\'aximo). Ahora, y teniendo en cuenta \textcite{Munck2002,Collier1999}, c\'omo debiera ser medida? Adopta una posici\'on y defi\'endela (1 p\'agina m\'aximo). Finalmente, y pensando en el clientelismo \parencite{Kitschelt2000,Auyero2000}, por qu\'e esta pr\'actica convive tan bien con la democracia, manteni\'endose en el tiempo? (desarrolla esta parte en profundidad en el resto de tu ensayo).
	\item Eval\'ua brevemente las teor\'ias de los or\'igenes de la democracia \parencite{Lipset1959,Przeworski1997,Moore:1966tn,Collier:1999rz,Boix:2003db,Acemoglu:1996rm} (resume cada teor\'ia en dos o tres oraciones como m\'aximo). A la luz de lo expuesto por \textcite{Haggard2012,Ansell:2014ty}, en qu\'e otras cr\'iticas (mencionadas en los textos y en las discusiones en clases/c\'apsulas) podemos pensar? (desarrolla esta parte en profundidad en el resto de tu ensayo).
	\item Explica brevemente la teor\'ia de \textcite{Downs:1957vg} (media p\'agina). Qu\'e condiciones hacen concluir a \textcite{Downs:1957vg} que los partidos tiendan a converger hacia el centro? Responde en funci\'on de \textcite{Boix:1999tj} (una p\'agina). Ahora, y pensando en los sistemas multi-partidistas y en la discusi\'on sostenida en clases, qu\'e factores podr\'ian causar una desestabilizaci\'on pol\'itica al mezclar multipartidismo con un presidencialismo fuerte? Responde a la luz de \textcite{Lijphart1990,Linz1985,Mainwaring1985} (desarrolla esta parte en profundidad en el resto de tu ensayo).
\end{enumerate}



% References
\newpage
\pagenumbering{Roman}
\setcounter{page}{1}
\printbibliography





\end{document}






